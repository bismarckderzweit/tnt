%*******************************************************
% Abstract
%*******************************************************
%\renewcommand{\abstractname}{Abstract}
\pdfbookmark[1]{Abstract}{Abstract}
\begingroup
\let\clearpage\relax
\let\cleardoublepage\relax
\let\cleardoublepage\relax

\chapter*{Abstract}
Learning to rank refers to machine learning techniques for training the model in a ranking task, and it is useful for many applications in Information Retrieval, Natural Language Processing, and Data Mining. But according to the current literature, as we know there are limited published papers explaining the LTR model effectively, while the models trained are often invisible to humans. Interpretabibility research on complex models is an important topic these years. For some areas such as stocks or diagnostics, we must deeply understand the internal mechanism of the model, otherwise we would rather sacrifice model performance than not be very sure of the model's decision.

In this thesis we analysis the interpretability of Learning to Rank models in a novel way. Basically, we propose a local feature attribution method that seeks to identify a small subset of input features as explanation to a ranking decision. With such subset we can recover a ranking $\pi ^{'}$, which has a very high similarity to $\pi$. In our research we defined two concepts called validity and completeness as matrics to measure the importance of features. We designed effective algorithms to seek validity and completeness of the features in subset. In general, if we list all the combinations of features, can always find the best subset as feature attribution, but for the LTR datasets with massive numbers, the number of total combinations is extremely big. Our proposed methods will greatly reduce the time cost for feature attribution. 

We tried a lot of experiments, the results reveal that with the compareation of agnostic explanation approache, which is shapley approach here. Our approach shows both accuracy and effectiveness. As a combination of validity optimized approach and completeness optimized approach, a new approach alpha, through harmonic factor $\alpha $ to make a trade of both methods is proposed, which can both optimize validity and completeness.


\vfill

%\pdfbookmark[1]{Zusammenfassung}{Zusammenfassung}
%\chapter*{Zusammenfassung}
%Kurze Zusammenfassung des Inhaltes in deutscher Sprache\dots


\endgroup			

\vfill